\section{Maticové rozklady}

\subsection*{LU rozklad}

Mějme matici $\mA\in T^{m,m}$. Pokud existuje dolní trojúhelníková matice $\mL$ s jedničkami na diagonále a horní trojúhelníková matice $\mU$, takové že $\mA=\mL\mU$, nazýváme tento součin \textbf{LU rozkladem}.

\subsection*{LU rozklad s řádkovou pivotací}

Zápis matice $\mA \in T^{m,m}$ jako součin
\[ \mP\mA = \mL\mU, \]
kde $\mP$ je nějaká permutační matice, $\mL$ dolní trojúhelníková matice s jedničkami na diagonále a $\mU$ horní trojúhelníková matice nazýváme \textbf{LU rozkladem s řádkovou pivotací}.

\subsection*{LU rozklad s částečnou pivotací}

LU rozklad s řádkovou pivotací matice $\mA \in \R^{m,m}$
\[ \mP\mA = \mL\mU, \]
nazveme \textbf{LU rozkladem s částečnou pivotací}, pokud pro všechna $i \geq j$ z $\hat m$ platí
\[ |\mL_{ij}|\leq 1\,. \]

\subsection*{Matice s ortonormálními sloupci}

Matici $\mQ\in\R^{m,n}$, $m \ge n$, nazýváme \textbf{maticí s ortonormálními sloupci}, pokud platí
\[ \mQ^T\mQ = \mE_n, \]
kde $\mE_n$ je jednotková matice z $\R^{n,n}$, neboli
\[ \mQ_{:i}^T\mQ_{:j} = \begin{cases} 0 & \text{ pro } i \neq j, \\ 1 & \text{ pro } i = j. \end{cases} \]

\subsection*{Ortogonální matice}

Čtvercovou matici $\mQ\in\R^{n,n}$ nazýváme \textbf{ortogonální}, pokud platí
\[ \mQ^T = \mQ^{-1}. \]

\subsection*{Transpozice ortogonální matice}

Je-li $\mQ$ ortogonální matice, je její transpozice $\mQ^T$ také ortogonální matice.


\begin{proof}[Důkaz]
	Z vlastnosti ortogonální matice a transpozice platí 
	\[ \mE=\mQ\mQ^T = \left(\mQ^T\right)^T\mQ^T  \text{ a } \mE=\mQ^T\mQ = \mQ^T\left(\mQ^T\right)^T. \]
	Tedy $\left(\mQ^T\right)^{-1}=\left(\mQ^T\right)^T$, a proto je $\mQ^T$ dle definice také ortogonální.
\end{proof}

\subsection*{Součin ortogonálních matic}

Pro ortogonální matice $\mQ_1, \mQ_2, \ldots, \mQ_k \in \R^{n,n}, n,k \in \N$, platí, že jejich součin je opět ortogonální matice, neboli
\[ (\mQ_1\mQ_2 \cdots \mQ_k)^T\mQ_1\mQ_2 \cdots \mQ_k = \mE. \]

\begin{proof}[Důkaz]
Tvrzení plyne z vlastnosti ortogonálních matic a z faktu, že transpozice součinu matic je součin transpozic těchto matic, ale v obráceném pořadí:
\[ (\mQ_1\mQ_2 \cdots\mQ_k)^T = \mQ_k^T\mQ_{k-1}^T \cdots\mQ_1^T\,. \]
Z toho a z vlastností ortogonálních matic získáváme
	\begin{gather*}
		(\mQ_1\mQ_2 \cdots\mQ_k)^T\mQ_1\mQ_2 \cdots\mQ_k = \mQ_k^T \cdots\mQ_2^T\mQ_1^T\mQ_1\mQ_2 \cdots\mQ_k = \\
		\mQ_k^T \cdots\mQ_2^T(\mQ_1^T\mQ_1)\mQ_2 \cdots\mQ_k =
		\mQ_k^T \cdots\mQ_2^T\,\mE\,\mQ_2 \cdots\mQ_k = \cdots = \mQ_k^T\mQ_k =\mE\,.
	\end{gather*}
\end{proof}

\subsection*{Ortogonální matice zachovávají standardní skalární součin}

Je-li $\mQ \in \R^{n,n}$ ortogonální matice, pak pro každé dva vektory $\vx, \vy \in \R^n$ platí
\[ (\mQ\vx)^T(\mQ\vy) = \vx^T \vy. \]

\begin{proof}[Důkaz]
	Důkaz je přímočarým použitím vlastností ON matic a vlastností standardního skalárního součinu
	\[ (\mQ\vx)^T(\mQ\vy) = \vx^T \mQ^T \mQ \vy = \vx^T \mE \vy = \vx^T \vy. \]
\end{proof}

\subsection*{Ortogonální matice zachovávají eukleidovskou normu}

Je-li $\mQ \in \R^{n,n}$ ortogonální matice, pak pro každý vektor $\vx \in \R^n$ platí
\[ \| \mQ\vx\|_2 = \|\vx\|_2. \]

\begin{proof}[Důkaz]
	Důkaz je přímočarým použitím vlastností eukleidovské normy a Tvrzení o OG maticích a skalárním součinu.
	\[ \| \mQ\vx\|_2^2 = (\mQ\vx)^T\mQ\vx = \vx^T\mQ^T\mQ\vx = \vx^T\vx = \| \vx \|_2^2. \]
\end{proof}

\subsection*{Determinant ortogonální matice}

Je-li $\mQ \in \R^{n,n}$ ortogonální matice, pak pro její determinant platí
\[ \det \mQ_i = \pm 1. \]

\begin{proof}[Důkaz]
	Důkaz plyne z vlastností determinantu součinu matic a transponované matice při aplikaci determinantu na obě strany rovnice ON matic,
	\begin{align*}
		\det \left( \mQ^T \mQ \right) &= \det \left(\mE \right),\\
		\det \left( \mQ^T \right) \det \left( \mQ \right) &= 1,\\
		\det \left(\mQ \right) \det \left(\mQ \right) &= 1,\\
		\left(\det \left(\mQ \right) \right)^2 &= 1,\\
		\det \left(\mQ \right) &= \pm 1.
	\end{align*}
\end{proof}

\subsection*{Vlastní čísla ortogonální matice}

Je-li $\mQ \in \R^{n,n}$ ortogonální matice, pak pro každé její vlastní číslo $\lambda$ platí
\[ |\lambda| = 1. \]

\begin{proof}[Důkaz]
	Důkaz plyne z Tvrzení o normě OG matic a homogenity násobení skalárním číslem použitého na rovnici pro vlastní čísla,
	\begin{align*}
		\mQ \vx &= \lambda \vx,\\
		\|\mQ \vx\|_2 &= \|\lambda \vx\|_2,\\
		\|\vx\|_2 &= |\lambda| \|\vx\|_2,\\
		1 &= |\lambda|.
	\end{align*}
\end{proof}

\subsection*{Redukovaný QR rozklad}

Mějme $m\geq n$ a matici $\mA \in \R^{m,n}$. Zápis této matice jako součin
\[ \mA = \hat{\mQ}\hat{\mR}, \]
kde $\hat{\mQ} \in \R^{m,n}$ je matice s ortonormálními sloupci a $\hat{\mR} \in \R^{n,n}$ je horní trojúhelníková matice, nazýváme \textbf{redukovaný QR rozklad}.

\subsection*{Úplný QR rozklad}

Mějme $m\geq n$ a matici $\mA \in \R^{m,n}$. Její zápis jako součin
\[ \mA = \mQ \mR, \]
kde $\mQ \in \R^{m,m}$ je ortogonální matice a $\mR \in \R^{m,n}$ je horní trojúhelníková matice, nazýváme \textbf{úplný (kompletní) QR rozklad}.

\subsection*{Věta o existenci QR rozkladu}

Každá matice $\mA \in \R^{m,n} (m \ge n)$ má úplný QR rozklad a tedy i redukovaný QR rozklad.

\subsection*{Věta o jednoznačnosti QR rozkladu}

Každá matice $\mA \in \R^{m,n} (m \ge n)$ s lineárně nezávislými sloupci má jednoznačný redukovaný QR rozklad $\mA=\hat{\mQ}\hat{\mR}$ splňující $r_{jj} > 0$.

\subsection*{Pozitivně definitní matice}

Mějme $\mA \in \R^{n,n}$.
Řekneme, že matice $\mA$ je \textbf{pozitivně definitní}, pokud
\[ \vx^T \mA \vx > 0 \quad \forall \vx \in \R^{n}, \vx \neq \theta. \]

\subsection*{Pozitivně semidefinitní matice}

Mějme $\mA \in \R^{n,n}$.
Řekneme, že matice $\mA$ je \textbf{pozitivně semidefinitní}, pokud
\[ \vx^T \mA \vx \geq 0 \quad \forall \vx \in \R^{n}. \]

\subsection*{Symetrické matice}

Matice $\mA \in \R^{n,n}$ je \textbf{symetrická}, jestliže je rovna svojí transpozici, tedy $\mA^T = \mA$.

\subsection*{Věta o symetrických maticích}

Buď $\mS \in \R^{n,n}$ symetrická matice. Potom platí následující:
\begin{enumerate}
	\item Matice $\mS$ je diagonalizovatelná a navíc lze vlastní vektory volit tak, že tvoří ortonormální bázi. Jinými slovy: existuje ortogonální matice $\mQ$ a diagonální matice $\mD$ tak, že
	\[ \mS = \mQ\mD\mQ^T. \]
	\item Všechna vlastní čísla matice $\mS$ jsou reálná.
	\item Je-li matice $\mS$ pozitivně semidefinitní, jsou vlastní čísla nezáporná.
	\item Je-li matice $\mS$ pozitivně definitní, jsou vlastní čísla kladná.
\end{enumerate}

\pagebreak