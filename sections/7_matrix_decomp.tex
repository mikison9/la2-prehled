\section{Maticové rozklady}

\subsection*{LU rozklad}

Mějme matici $\mA\in T^{m,m}$. Pokud existuje dolní trojúhelníková matice $\mL$
s jedničkami na diagonále a horní trojúhelníková matice $\mU$, takové že
$\mA=\mL\mU$, nazýváme tento součin \textbf{LU rozkladem}.

\subsection*{LU rozklad s řádkovou pivotací}

Zápis matice $\mA \in T^{m,m}$ jako součin
\[ \mP\mA = \mL\mU, \]
kde $\mP$ je nějaká permutační matice, $\mL$ dolní trojúhelníková matice s
jedničkami na diagonále a $\mU$ horní trojúhelníková matice nazýváme \textbf{LU
	rozkladem s řádkovou pivotací}.

\subsection*{LU rozklad s částečnou pivotací}

LU rozklad s řádkovou pivotací matice $\mA \in \R^{m,m}$
\[ \mP\mA = \mL\mU, \]
nazveme \textbf{LU rozkladem s částečnou pivotací}, pokud pro všechna $i \geq
	j$ z $\hat m$ platí
\[ |\mL_{ij}|\leq 1\,. \]

\subsection*{Matice s ortonormálními sloupci}

Matici $\mQ\in\R^{m,n}$, $m \ge n$, nazýváme \textbf{maticí s ortonormálními
	sloupci}, pokud platí
\[ \mQ^T\mQ = \mE_n, \]
kde $\mE_n$ je jednotková matice z $\R^{n,n}$, neboli
\[ \mQ_{:i}^T\mQ_{:j} = \begin{cases} 0 & \text{ pro } i \neq j, \\ 1 & \text{ pro } i = j. \end{cases} \]

\subsection*{Ortogonální matice}

Čtvercovou matici $\mQ\in\R^{n,n}$ nazýváme \textbf{ortogonální}, pokud platí
\[ \mQ^T = \mQ^{-1}. \]

\subsection*{Transpozice ortogonální matice}

Je-li $\mQ$ ortogonální matice, je její transpozice $\mQ^T$ také ortogonální
matice.

\begin{proof}[Důkaz]
	Z vlastnosti ortogonální matice a transpozice platí
	\[ \mE=\mQ\mQ^T = \left(\mQ^T\right)^T\mQ^T  \text{ a } \mE=\mQ^T\mQ = \mQ^T\left(\mQ^T\right)^T. \]
	Tedy $\left(\mQ^T\right)^{-1}=\left(\mQ^T\right)^T$, a proto je $\mQ^T$ dle
	definice také ortogonální.
\end{proof}

\subsection*{Součin ortogonálních matic}

Pro ortogonální matice $\mQ_1, \mQ_2, \ldots, \mQ_k \in \R^{n,n}, n,k \in \N$,
platí, že jejich součin je opět ortogonální matice, neboli
\[ (\mQ_1\mQ_2 \cdots \mQ_k)^T\mQ_1\mQ_2 \cdots \mQ_k = \mE. \]

\begin{proof}[Důkaz]
	Tvrzení plyne z vlastnosti ortogonálních matic a z faktu, že transpozice součinu matic je součin transpozic těchto matic, ale v obráceném pořadí:
	\[ (\mQ_1\mQ_2 \cdots\mQ_k)^T = \mQ_k^T\mQ_{k-1}^T \cdots\mQ_1^T\,. \]
	Z toho a z vlastností ortogonálních matic získáváme
	\begin{gather*}
		(\mQ_1\mQ_2 \cdots\mQ_k)^T\mQ_1\mQ_2 \cdots\mQ_k = \mQ_k^T \cdots\mQ_2^T\mQ_1^T\mQ_1\mQ_2 \cdots\mQ_k = \\
		\mQ_k^T \cdots\mQ_2^T(\mQ_1^T\mQ_1)\mQ_2 \cdots\mQ_k =
		\mQ_k^T \cdots\mQ_2^T\,\mE\,\mQ_2 \cdots\mQ_k = \cdots = \mQ_k^T\mQ_k =\mE\,.
	\end{gather*}
\end{proof}

\subsection*{Ortogonální matice zachovávají standardní skalární součin}

Je-li $\mQ \in \R^{n,n}$ ortogonální matice, pak pro každé dva vektory $\vx,
	\vy \in \R^n$ platí
\[ (\mQ\vx)^T(\mQ\vy) = \vx^T \vy. \]

\begin{proof}[Důkaz]
	Důkaz je přímočarým použitím vlastností ON matic a vlastností standardního skalárního součinu
	\[ (\mQ\vx)^T(\mQ\vy) = \vx^T \mQ^T \mQ \vy = \vx^T \mE \vy = \vx^T \vy. \]
\end{proof}

\subsection*{Ortogonální matice zachovávají eukleidovskou normu}

Je-li $\mQ \in \R^{n,n}$ ortogonální matice, pak pro každý vektor $\vx \in
	\R^n$ platí
\[ \| \mQ\vx\|_2 = \|\vx\|_2. \]

\begin{proof}[Důkaz]
	Důkaz je přímočarým použitím vlastností eukleidovské normy a Tvrzení o OG maticích a skalárním součinu.
	\[ \| \mQ\vx\|_2^2 = (\mQ\vx)^T\mQ\vx = \vx^T\mQ^T\mQ\vx = \vx^T\vx = \| \vx \|_2^2. \]
\end{proof}

\subsection*{Determinant ortogonální matice}

Je-li $\mQ \in \R^{n,n}$ ortogonální matice, pak pro její determinant platí
\[ \det \mQ_i = \pm 1. \]

\begin{proof}[Důkaz]
	Důkaz plyne z vlastností determinantu součinu matic a transponované matice při aplikaci determinantu na obě strany rovnice ON matic,
	\begin{align*}
		\det \left( \mQ^T \mQ \right)                     & = \det \left(\mE \right), \\
		\det \left( \mQ^T \right) \det \left( \mQ \right) & = 1,                      \\
		\det \left(\mQ \right) \det \left(\mQ \right)     & = 1,                      \\
		\left(\det \left(\mQ \right) \right)^2            & = 1,                      \\
		\det \left(\mQ \right)                            & = \pm 1.
	\end{align*}
\end{proof}

\subsection*{Vlastní čísla ortogonální matice}

Je-li $\mQ \in \R^{n,n}$ ortogonální matice, pak pro každé její vlastní číslo
$\lambda$ platí
\[ |\lambda| = 1. \]

\begin{proof}[Důkaz]
	Důkaz plyne z Tvrzení o normě OG matic a homogenity násobení skalárním číslem použitého na rovnici pro vlastní čísla,
	\begin{align*}
		\mQ \vx       & = \lambda \vx,         \\
		\|\mQ \vx\|_2 & = \|\lambda \vx\|_2,   \\
		\|\vx\|_2     & = |\lambda| \|\vx\|_2, \\
		1             & = |\lambda|.
	\end{align*}
\end{proof}

\subsection*{Redukovaný QR rozklad}

Mějme $m\geq n$ a matici $\mA \in \R^{m,n}$. Zápis této matice jako součin
\[ \mA = \hat{\mQ}\hat{\mR}, \]
kde $\hat{\mQ} \in \R^{m,n}$ je matice s ortonormálními sloupci a $\hat{\mR}
	\in \R^{n,n}$ je horní trojúhelníková matice, nazýváme \textbf{redukovaný QR
	rozklad}.

\subsection*{Úplný QR rozklad}

Mějme $m\geq n$ a matici $\mA \in \R^{m,n}$. Její zápis jako součin
\[ \mA = \mQ \mR, \]
kde $\mQ \in \R^{m,m}$ je ortogonální matice a $\mR \in \R^{m,n}$ je horní
trojúhelníková matice, nazýváme \textbf{úplný (kompletní) QR rozklad}.

\subsection*{Souvislost QR rozkladu a GSO}

\begin{equation*}
	  \pmat{c|c|c|c}{ &     &       &     \\ &     &       &     \\ \va_1 & \va_2 & \dots & \va_n \\ &     &       &     \\ &     &       & }
	=
	\pmat{c|c|c|c}{ &     &       &     \\ &     &       &     \\ \vq_1 & \vq_2 & \dots & \vq_n \\ &     &       &     \\ &     &       & }
	\pmat{cccc}{ r_{11} & r_{12} & \dots  & r_{1n} \\ & r_{22} & \dots  & r_{2n} \\ &        & \ddots & \vdots \\ &        &        & r_{nn} }\,.
\end{equation*}

\subsection*{Věta o existenci QR rozkladu}

Každá matice $\mA \in \R^{m,n} (m \ge n)$ má úplný QR rozklad a tedy i
redukovaný QR rozklad.

\subsection*{Věta o jednoznačnosti QR rozkladu}

Každá matice $\mA \in \R^{m,n} (m \ge n)$ s lineárně nezávislými sloupci má
jednoznačný redukovaný QR rozklad $\mA=\hat{\mQ}\hat{\mR}$ splňující $r_{jj} >
	0$.

\subsection*{Householderova triangularizace}

Máme-li QR rozklad

\[ \mA = \mQ\mR, \]

\noindent tak vlastně platí

\[ \mQ^T \mA = \mR, \]

\noindent Chceme najít ortogonální matice $\mQ_1, \mQ_2, \ldots, \mQ_k$, které postupně zvyšují počet nul pod diagonálou:

\[
\underset{\mA}{ \left({\begin{array}{*{20}c} \circ    & \circ    & \circ    \\ \circ    & \circ    & \circ    \\ \circ    & \circ    & \circ    \\ \circ    & \circ    & \circ    \\ \circ    & \circ    & \circ    \\ \end{array}} \right) }
\xrightarrow{\mQ_1}
\underset{\mQ_1 \mA}{ \left({\begin{array}{*{20}c} \bullet  & \bullet  & \bullet  \\ \mathbf{0}  & \bullet  & \bullet  \\ \mathbf{0}  & \bullet  & \bullet  \\ \mathbf{0}  & \bullet  & \bullet  \\ \mathbf{0}  & \bullet  & \bullet  \\ \end{array}} \right) }
\xrightarrow{\mQ_2}
\underset{\mQ_2 \mQ_1 \mA}{ \left({\begin{array}{*{20}c} \circ    & \circ    & \circ    \\ \mathbf{0}  & \bullet  & \bullet  \\ \mathbf{0}  & \mathbf{0}  & \bullet  \\ \mathbf{0}  & \mathbf{0}  & \bullet  \\ \mathbf{0}  & \mathbf{0}  & \bullet  \\ \end{array}} \right) }
\xrightarrow{\mQ_3}
\underset{\mQ_3 \mQ_2 \mQ_1 \mA}{ \left({\begin{array}{*{20}c} \circ    & \circ    & \circ    \\ 0       & \circ    & \circ    \\ \mathbf{0}  & \mathbf{0}  & \bullet  \\ \mathbf{0}  & \mathbf{0}  & \mathbf{0}  \\ \mathbf{0}  & \mathbf{0}  & \mathbf{0}  \\ \end{array}} \right) }\,.
\] 

\subsection*{Householderův reflektor}

Hledaná ortogonální matice zachová prvních $k-1$ řádků, bude mít tvar

\[ \mQ_k = \left({ \begin{array}{*{20}c} \mE_{k-1}  &    \\ &  \mF \\ \end{array}} \right). \]

\noindent Označme si sloupec matice od diagonály dolů, který chceme vynulovat, jako $\vx$.
Hledáme $\mF$ tak, aby

\[ \vx = \left({ \begin{array}{*{20}c} x_{k}   \\ x_{k+1} \\ \vdots  \\ x_{n}   \\ \end{array}} \right)
\quad
\xrightarrow{\mF}
\quad
\mF \vx = \left({ \begin{array}{*{20}c} \|\vx\| \\ 0 \\ \vdots \\ 0 \\ \end{array}} \right) = \|\vx\| \ve_1. 
\]

\noindent Rozdíl těchto vektorů označme jako

\[ \vv = \|\vx\|\ve_1 -\vx. \]

\noindent Definujme podprostor $H$ jako ortogonální doplněk vektoru $\vv$. Projekci $\vx$ na přímku určenou vektorem $\vv$ získáme vzorečkem

\[ \proj_{\vv}\vx = \frac{\vx^T \vv}{\vv^T \vv} \vv \]

\noindent a projekci na podprostor $H$

\[ \proj_{H}\vx = \vx - \proj_{\vv} \vx = \vx - \frac{\vx^T \vv}{\vv^T \vv} \vv = \vx -\vv \frac{\vv^T \vx}{\vv^T \vv} = \left(\mE - \frac{\vv\vv^T }{\vv^T \vv}\right)\vx. \]

\noindent Pokud chceme získat zrcadlení, projekci na vektor $\vv$ musíme odečíst dvakrát, tedy

\[ \mF\vx = \vx - 2\proj_{\vv} \vx = \vx - 2\frac{\vx^T \vv}{\vv^T \vv} \vv = \vx -2\vv \frac{\vv^T \vx}{\vv^T \vv} = \left(\mE - 2\frac{\vv\vv^T }{\vv^T \vv}\right)\vx. \]

\noindent Hledaná matice zrcadlení tedy bude

\[ \mF = \mE - \clr{blue}{2}\frac{\vv\vv^T }{\vv^T \vv}. \]

\noindent To ale není jediná možnost, jak sloupec vynulovat.
Druhá možnost odpovídá zobrazení vektoru $\vx$ na vektor $-\|\vx\|\ve_1$.
Druhý reflektor je tedy

\[ \vw = - \|\vx\|\ve_1 - \vx. \]

\noindent Z důvodu numerické chyby vybereme ten, který bude mít větší délku, jako

\[ \vv = - \sgn(x_1) \|\vx\|\ve_1 - \vx \]

\noindent Konečně si uvědomme, že na orientaci nezáleží, a zbavíme se tak znaménka

\[ \vv = \sgn(x_1) \|\vx\|\ve_1 + \vx \]

\begin{figure*}[ht]
	\centering
	\begin{tikzpicture}
    % x-axis
\draw (-3,0) -- (3,0);
% y-axis
\draw (0,-1) -- (0,2);
% vector x, angle 40 degs
\draw[vector] (0,0) -- ({2*cos(40)},{2*sin(40)}) node[anchor=south east] {$\vx$};
% mirror axis, angle 20 degs
\draw[mirror] ({-3*cos(20)},{-3*sin(20)}) node[anchor=south west,yshift = 2pt,xshift = -5pt] {$H^+$} -- ({3*cos(20)},{3*sin(20)});
% mirror axis, angle 110 degs
\draw[mirror] ({-1*cos(110)},{-1*sin(110)}) node[anchor=south west,yshift = 5pt,xshift = -3pt] {$H^-$} -- ({2*cos(110)},{2*sin(110)});
% Householder reflector
\draw[vector,blue] ({2*cos(40)},{2*sin(40)}) -- (2,0) node[anchor=south west,yshift = 8pt,xshift = -5pt] {$\vv$};
% 2nd Householder reflector
\draw[vector,blue] ({2*cos(40)},{2*sin(40)}) -- (-2,0) node[anchor=south west,yshift = 12pt,xshift = 20pt] {$\vw$};
% target vector
\draw[vector] (0,0) -- (2,0) node[anchor=north west,xshift = -15pt] {$+\|\vx\|\ve_1$};
% second target vector
\draw[vector] (0,0) -- (-2,0) node[anchor=north west,xshift = -20pt] {$-\|\vx\|\ve_1$};
	\end{tikzpicture}
\end{figure*}

\subsection*{Givensova triangularizace}

Další možnost jak pomocí OG matice zobrazit vektor do směru je rotace. Této matici se říká \textbf{Givensova rotace}.

\begin{figure*}[ht]
	\centering
	\begin{tikzpicture}
		% x-axis
		\draw (-3,0) -- (3,0);
		% y-axis
		\draw (0,-1) -- (0,2);
		% vector x, angle 40 degs
		\draw[vector] (0,0) -- ({2*cos(40)},{2*sin(40)}) node[anchor=south east] {$\vx$};
		% target vector
		\draw[vector] (0,0) -- (2,0) node[anchor=north west,xshift = -15pt] {$\mG \vx$};
		% arc for the rotation
		\draw[vector,blue,dashed] ({2*cos(40)},{2*sin(40)}) arc (40:0:2);
		% arc for the angle
		\draw[->,>=stealth] (0.8,0) arc (0:40:0.8) node[anchor=north west,xshift = 5pt,yshift = 3pt] {$\theta$};
	\end{tikzpicture}
	\caption*{Givensova rotace}
\end{figure*}

\noindent V prostoru $\R^m$, $c = \cos\theta = \frac{x_i}{\sqrt{x_i^2 + x_j^2}}$, $s = \sin\theta = \frac{x_j}{\sqrt{x_i^2 + x_j^2}}$:

\[ \mG(x_i,x_j) = \begin{pmatrix} 1      &        &          &        &         &        &    \\ & \ddots &          &        &         &        &    \\ &        &  c       &        &  s      &        &    \\ &        &          & \ddots &         &        &    \\ &        &  -s      &        &  c      &        &    \\ &        &          &        &         & \ddots &    \\ &        &          &        &         &        & 1 \end{pmatrix} \]

\noindent Givensova triangularizace postupuje následovně.

\begin{gather*}
	  \underset{\mA}{\left({\begin{array}{*{20}c} \circ    & \circ    & \circ    \\ \circ    & \circ    & \circ    \\ \circ    & \circ    & \circ    \\ \circ    & \circ    & \circ    \\ \circ    & \circ    & \circ    \\ \end{array}} \right) }
	\xrightarrow{\mQ_1}
	\underset{\mQ_1 \mA}{ \left({\begin{array}{*{20}c} \bullet  & \bullet  & \bullet  \\ \mathbf{0}  & \bullet  & \bullet  \\ \circ    & \circ    & \circ    \\ \circ    & \circ    & \circ    \\ \circ    & \circ    & \circ    \\ \end{array}} \right) }
	\xrightarrow{\mQ_2}
	\underset{\mQ_2 \mQ_1 \mA}{ \left({\begin{array}{*{20}c} \bullet  & \bullet  & \bullet  \\ 0        & \circ    & \circ    \\ \mathbf{0}  & \bullet  & \bullet  \\ \circ    & \circ    & \circ    \\ \circ    & \circ    & \circ    \\ \end{array}} \right) }
	\xrightarrow{\mQ_3}
	\underset{\mQ_3 \mQ_2 \mQ_1 \mA}{ \left({\begin{array}{*{20}c} \bullet  & \bullet  & \bullet  \\ 0       & \circ    & \circ    \\ 0       & \circ    & \circ    \\ \mathbf{0}  & \bullet  & \bullet  \\ \circ    & \circ    & \circ    \\ \end{array}} \right) }
	\\
	\xrightarrow{\mQ_4}
	\underset{\mQ_4 \mQ_3 \mQ_2 \mQ_1 \mA}{ \left({\begin{array}{*{20}c} \bullet  & \bullet  & \bullet  \\ 0       & \circ    & \circ    \\ 0       & \circ    & \circ    \\ 0       & \circ    & \circ    \\ \mathbf{0}  & \bullet  & \bullet  \\ \end{array}} \right) }
	\xrightarrow{\mQ_5}
	\underset{\mQ_5 \mQ_4 \mQ_3 \mQ_2 \mQ_1 \mA}{ \left({\begin{array}{*{20}c} \circ    & \circ    & \circ    \\ \mathbf{0}  & \bullet  & \bullet  \\ \mathbf{0}  & \mathbf{0}  & \bullet  \\ 0       & \circ    & \circ    \\ 0       & \circ    & \circ    \\ \end{array}} \right) }
	\xrightarrow{\mQ_6}
	\underset{\mQ_6 \mQ_5 \mQ_4 \mQ_3 \mQ_2 \mQ_1 \mA}{ \left({\begin{array}{*{20}c} \circ    & \circ    & \circ    \\ \mathbf{0}  & \bullet  & \bullet  \\ 0       & 0        & \circ    \\ \mathbf{0}  & \mathbf{0}  & \bullet  \\ 0       & \circ    & \circ    \\ \end{array}} \right) }
	\ldots
	\xrightarrow{\mQ_k}
	\underset{\mQ_k \dots \mQ_2 \mQ_1 \mA}{ \left({\begin{array}{*{20}c} \circ    & \circ    & \circ    \\ 0       & \circ    & \circ    \\ \mathbf{0}  & \mathbf{0}  & \bullet  \\ 0       & 0        & 0        \\ \mathbf{0}  & \mathbf{0}  & \mathbf{0}  \\ \end{array}} \right) }
\end{gather*}

\begin{algorithm*}
	\caption{Givensova triangularizace}
	\begin{algorithmic}
		\FOR{$i = 1, \dots, n$}
		\FOR{$j = i+1, \dots, m$}
		\STATE Najdi $\mG(a_{i,i},a_{j,i})$
		\STATE $\mA_{[i,j],i:n} = \mG \mA_{[i,j],i:n}$
		\ENDFOR
		\ENDFOR
	\end{algorithmic}
\end{algorithm*}

\subsection*{Pozitivně definitní matice}

Mějme $\mA \in \R^{n,n}$. Řekneme, že matice $\mA$ je \textbf{pozitivně
	definitní}, pokud
\[ \vx^T \mA \vx > 0 \quad \forall \vx \in \R^{n}, \vx \neq \theta. \]

\subsection*{Pozitivně semidefinitní matice}

Mějme $\mA \in \R^{n,n}$. Řekneme, že matice $\mA$ je \textbf{pozitivně
	semidefinitní}, pokud
\[ \vx^T \mA \vx \geq 0 \quad \forall \vx \in \R^{n}. \]

\subsection*{Symetrické matice}

Matice $\mA \in \R^{n,n}$ je \textbf{symetrická}, jestliže je rovna svojí
transpozici, tedy $\mA^T = \mA$.

\subsection*{Věta o symetrických maticích}

Buď $\mS \in \R^{n,n}$ symetrická matice. Potom platí následující:
\begin{enumerate}
	\item Matice $\mS$ je diagonalizovatelná a navíc lze vlastní vektory volit tak, že
	      tvoří ortonormální bázi. Jinými slovy: existuje ortogonální matice $\mQ$ a
	      diagonální matice $\mD$ tak, že
	      \[ \mS = \mQ\mD\mQ^T. \]
	\item Všechna vlastní čísla matice $\mS$ jsou reálná.
	\item Je-li matice $\mS$ pozitivně semidefinitní, jsou vlastní čísla nezáporná.
	\item Je-li matice $\mS$ pozitivně definitní, jsou vlastní čísla kladná.
\end{enumerate}

\subsection*{Rayleighův podíl}

\textbf{Rayleighův podíl} vektoru $\vy \in \R^{n}, \vy \ne \theta$ je číslo

\[ r(\vy) = \frac{\vy^T \mA \vy}{\vy^T\vy}. \]

\noindent Nechť $\vx$ je vlastní vektor matice $\mA$ příslušný vlastnímu číslu $\lambda$,
neboli platí

\[ \mA \vx = \lambda \vx. \]

\noindent Pak platí $r(\vx) = \lambda$.

\subsection*{Mocninná metoda}

Matice $\mA$ \textbf{musí být symetrická}.

\vspace{0.5em}

\begin{algorithmic}
	\STATE $\vv^{(0)} = \ \mbox{nějaký vektor takový, že} \ \|\vv^{(0)}\| = 1$
	\FOR {$k = 1, 2, \ldots$ }
	\STATE $\vw = \mA\vv^{(k-1)}$
	\STATE $\vv^{(k)} = \vw / \|\vw\|$
	\STATE $\lambda^{(k)} = (\vv^{(k)})^T \mA\vv^{(k)}$
	\ENDFOR
\end{algorithmic}

\vspace{0.5em}

\noindent $\lambda^{(k)}$ konverguje k dominantnímu (největšímu) vl. číslu $\mA$ a $\vv^{(k)}$ konverguje k příslušnému vl. vektoru.

\subsection*{QR algoritmus pro výpočet vlastních čísel}

Matice $\mA$ \textbf{musí být symetrická}.

\vspace{0.5em}

\begin{algorithmic}
	\STATE $\mA^{(0)} = \mA$
	\FOR {$k = 1, 2, \ldots$ }
	\STATE $\mQ^{(k)}\mR^{(k)} = \mA^{(k-1)}$
	\STATE $\mA^{(k)} = \mR^{(k)}\mQ^{(k)}$
	\ENDFOR
\end{algorithmic}

\vspace{0.5em}

\noindent Pro symetrické matice posloupnost $\mA^{(0)}, \mA^{(1)}, \mA^{(2)}, \dots$ konverguje k diagonální matici, která je \emph{podobná} $\mA$ a tudíž má stejná vlastní čísla.

\subsection*{Příprava matice pomocí ortogonálních transformací}


Najdeme tedy Householderův reflektor tak, že

\[ \mQ_1 = \left( {\begin{array}{*{20}c} 1 &    \\ & \mF \end{array} } \right), \]

\noindent dostáváme

\[ \mQ_1\mA\mQ_1 = \underbrace{\pmat{cccc}{ \circ &  \circ  &  \circ  &  \circ \\ \alpha_1 &  \bullet  &  \bullet  &  \bullet \\ 0 &  \bullet  &  \bullet &  \bullet  \\ 0 &  \bullet  &  \bullet &  \bullet \\}}_{\mQ_1\mA} \mQ_1 = \underbrace{\pmat{cccc}{ \circ    &  \bullet  &  \bullet  &  \bullet \\ \alpha_1 &  \bullet  &  \bullet  &  \bullet \\ 0        &  \bullet  &  \bullet  &  \bullet \\ 0        &  \bullet  &  \bullet  &  \bullet \\ }}_{\mQ_1\mA\mQ_1}. \]

\noindent Analogicky vytvoříme matici $\mQ_2$. Dostáváme

\[ \mQ_2\mQ_1\mA\mQ_1\mQ_2 = \mQ_2 \underbrace{\pmat{cccc}{ \circ    &  \circ    &  \circ    &  \circ \\ \alpha_1 &  \circ    &  \circ    &  \circ \\ 0        &  \circ    &  \circ    &  \circ \\ 0        &  \circ    &  \circ    &  \circ \\}}_{\mQ_1\mA\mQ_1} \mQ_2 = \underbrace{\pmat{cccc}{ \circ    &  \circ    &  \bullet  &  \bullet \\ \alpha_1 &  \circ    &  \bullet  &  \bullet \\ 0        &  \alpha_2 &  \bullet  &  \bullet \\ 0        &  0        &  \bullet  &  \bullet \\}}_{\mQ_2\mQ_1\mA\mQ_1\mQ_2} \]

\noindent Tímto postupem jsme dospěli k matici v tzv. \emph{Hessenbergově tvaru}.

\noindent Matici $\mA$ jsme upravili jako

\[ \mQ^T\mA\mQ = \mathbf{H}, \]

\noindent neboli jsme našli faktorizaci ve tvaru

\[ \mA = \mQ \mathbf{H} \mQ^T. \]

\noindent Z tohoto zápisu vyplývá, že matice $\mA$ a $\mathbf{H}$ jsou si podobné a tudíž mají stejná vlastní čísla.

\subsection*{Dvě fáze výpočtu vlastních čísel}

\begin{enumerate}
	\item Převedení matice do Hessenbergova tvaru pomocí Householderovy redukce (přímý algoritmus).
	\item Převedení matice z Hessenbergova tvaru do horního trojúhelníkového tvaru pomocí iteračního algoritmu, např. QR algoritmu.
\end{enumerate}

\subsection*{Redukovaný SVD rozklad}

Redukovaným SVD rozkladem matice $\mA\in\R^{m,n}$, $m \ge n$, se myslí její zápis jako součin matic

\[ \mA = \hat{\mU} \hat{\mSigma} \mV^T, \]

\noindent kde $\hat{\mU} \in \R^{m,n}$ je matice s ortonormálními sloupci, $\hat{\mSigma} \in \R^{n,n}$ je diagonální matice, a $\mV \in \R^{n,n}$ je ortogonální matice.
Na diagonále $\hat{\mSigma}$ je $p = \min(m,n)$ seřazených \emph{singulárních čísel} $\sigma_1 \ge \sigma_2 \ge \dots \ge \sigma_p \ge 0$.

\subsection*{Úplný SVD rozklad}

Úplným (kompletním) SVD rozkladem matice $\mA\in\R^{m,n}$ se myslí její zápis jako součin matic

\[ \mA = \mU \mSigma \mV^T, \]

\noindent kde $\mU \in \R^{m,m}$ je ortogonální matice, $\mSigma \in \R^{m,n}$ je diagonální matice, a $\mV \in \R^{n,n}$ je ortogonální matice.
Na diagonále $\mSigma$ je $p = \min(m,n)$ seřazených \emph{singulárních čísel} $\sigma_1 \ge \sigma_2 \ge \dots \ge \sigma_p \ge 0$.

\subsection*{Existence a jednoznačnost SVD rozkladu}

Každá matice $\mA \in \R^{m,n}$ má (úplný) SVD rozklad. Singulární čísla $\{\sigma_j\}$ jsou určena jednoznačně a pokud je $\mA$ čtvercová a singulární čísla $\{\sigma_j\}$ jsou různá, levé a pravé singulární vektory $\{\vu_j\}$ a $\{\vv_j\}$ jsou určeny jednoznačně až na koeficient $\pm 1$.

\subsection*{SVD rozklad vs. rozklad pomocí vlastních čísel}

Nenulová singulární čísla $\mA \in \R^{m,n}$ jsou rovna odmocninám z nenulových vlastních čísel $\mA^T\mA$, resp. $\mA\mA^T$.

\subsection*{SVD rozklad a vlastnosti matic}

Nechť $\mA \in \R^{m,n}$, $p = \min(m,n)$, a $r \le p$ je počet nenulových singulárních čísel $\mA$. Pak platí následující tvrzení.

\begin{enumerate}
	\item $h(\mA) = r$
	\item Lineární obal sloupců $\mA$ je roven $\langle \vu_1,\dots,\vu_r\rangle$.
	Podprostor řešení soustavy $\{\vx \in \R^n, \mA \vx = \theta\}$ je roven $\langle \vv_{r+1},\dots,\vv_{n}\rangle$.
	\item Pro každou matici $\mA \in \R^{m,n}$ a ortogonální matici $\mQ \in \R^{m,m}$ platí
		\[ \|\mQ\mA\|_2 = \|\mA\|_2, \quad \|\mQ\mA\|_F = \|\mA\|_F. \]
	\item \[ \|\mA\|_2 = \sigma_1 \quad \mbox{a} \quad \|\mA\|_F = \sqrt{\sigma_1^2 + \sigma_2^2 + \dots + \sigma_r^2}. \]
\end{enumerate}

\pagebreak