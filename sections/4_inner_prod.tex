\section{Skalární součin}

\subsection*{Skalární součin}

Buď $V$ VP nad $\clr{red}{T=\R}$ nebo $\clr{red}{\CC}$. Zobrazení $\langle
    \cdot\mid\cdot\rangle :$ $V\times V \rightarrow T$ nazýváme (obecný)
\textbf{skalární součin}, platí-li pro všechny vektory $\vx,\vy,\vz\in V$ a
každý skalár $\alpha\in T$ následující axiomy:

\begin{enumerate}
    \item Zobrazení je lineární v druhém argumentu, tj.

          \[ \langle \vx\mid \vy +\vz\rangle =\langle \vx\mid \vy\rangle +\langle \vx\mid \vz\rangle\quad \text{a}\quad
              \langle \vx\mid \alpha \vy\rangle=\alpha\, \langle \vx\mid \vy\rangle\,. \]

    \item Platí tzv. \emph{Hermitovská symetrie}:

          \[ \langle \vx \mid \vy\rangle =\overline{\langle \vy\mid \vx \rangle}\,. \]

    \item Zobrazení je pozitivně definitní, tzn.

          \[ \langle \vx \mid \vx \rangle \geq0 \quad \text{ a zároveň }\quad
              \big(\,\langle \vx \mid \vx \rangle =0 \Leftrightarrow \vx =\theta\,\big). \]
\end{enumerate}

Dvojici $(V, \langle \cdot \mid \cdot \rangle)$ nazýváme \textbf{(vektorovým)
    prostorem se skalárním součinem} nebo zkráceně jako \textbf{prehilbertův
    prostor} a značíme $\mathcal{H}$.

\subsection*{Základní vlastnosti skalárního součinu}

Pro libovolné $\vx,\vy,\vz\in\mathcal{H}$ a $\alpha\in T$ platí

\begin{enumerate}
    \item Sdružená linearita v prvním argumentu:
          \[ \langle \vx +\vy\mid \vz\rangle=\langle \vx\mid \vz\rangle+\langle \vy\mid \vz\rangle\quad \text{a}\quad
              \langle \alpha \vx \mid \vz\rangle=\overline{\alpha}\,\langle \vx\mid \vz\rangle\,. \]

    \item Skalární součin s nulovým vektorem je nula:
          \[ \langle \vx\mid \theta\rangle=\langle \theta\mid \vx\rangle=0\,. \]
\end{enumerate}

\subsection*{Standardní skalární součin}

Na $T^{n}$ definujeme skalární součin předpisem

\[ \vx\centerdot \vy:=\sum_{j=1}^{n}\overline{x_j}\cdot y_{j}\,, \]

\noindent kde $\vx=(x_1,\dots,x_n)$, $\vy=(y_1,\dots,y_n)$ jsou vektory z $T^n$. Tento skalární součin nazýváme \textbf{standardním skalárním součinem}.

\subsection*{Norma, vzdálenost}

Mějme $V$ VP nad $\R$ nebo $\CC$. Zobrazení $\|\cdot\|:V\to \R$ nazýváme
\textbf{norma}, pokud pro libovolné $\vx, \vy \in V$ a $\alpha\in T$ platí:

\begin{enumerate}
    \item norma je vždy nezáporná:

          \[ \|\vx\|\geq0 \,, \]

    \item pouze nulový vektor má nulovou normu:

          \[ \|\vx\|=0 \Leftrightarrow \vx=\theta\,, \]

    \item norma je homogenní v absolutní hodnotě:

          \[ \|\alpha \vx\|=|\alpha|\cdot\|\vx\|\,, \]

    \item platí \textbf{trojúhelníková nerovnost}:

          \[ \|\vx+\vy\|\leq\|\vx\|+\|\vy\|\,. \]

\end{enumerate}

\noindent Pro $\vx, \vy\in V$ číslo $\|\vx\|\in \R$ nazýváme \textbf{velikostí vektoru} $\vx$ a číslo $d(x,y):=\|\vx-\vy\|$ nazýváme \textbf{vzdáleností vektorů} $\vx$
a $\vy$.

\subsection*{Norma indukovaná skalárním součinem}

Ve vektorovém prostoru $\mathcal H$ se skalárním součinem
$\langle\cdot\mid\cdot\rangle$ definujme zobrazení $\|\cdot\|: \mathcal H\to
    \R$ pro $\vx\in \mathcal H$ předpisem

\[ \|\vx\|:=\sqrt{\langle\vx\mid\vx\rangle}\,. \]

\noindent Toto zobrazení nazýváme \textbf{normou indukovanou skalárním součinem}.

\subsection*{Věta o korektnosti indukované normy a Schwarzova nerovnost}

Zobrazení definované v Definici .reference:dfn-indukovana-norma je normou a
splňuje tzv. \textbf{Schwarzovu nerovnost}:

Pro každé $\vx$, $\vy$ z $\mathcal H$ platí

\[ |\langle\vx\mid\vy\rangle|\leq \|\vx\|\cdot\|\vy\|\,. \]

\subsection*{Eukleidovská norma}

Norma indukovaná standardním součinem se nazývá \textbf{eukleidovská norma}.

Pro $\vx=(x_1,\dots,x_n)\in T^n$ je eukleidovská norma rovna

\[ \|\vx\|=\sqrt{\,\overline{x_1}\,x_1+\dots+\overline{x_n}\,x_n}=\sqrt{|x_1|^2+\dots+|x_n|^2}\,. \]

\subsection*{$p$-norma}

Na $T^n$ definujeme pro $p\in \langle 1,\infty\rangle$ tzv. \textbf{$p$-normu}
předpisem: pro $\vx=(x_1,\dots,x_n)\in T^n$ položíme

\[ \|\vx\|_p=\begin{cases}
        \Big(|x_1|^p+\dots+|x_n|^p\Big)^{\frac{1}{p}}\, & \text{ pro }p < \infty\,, \\
        \max\big\{|x_1|,\dots,|x_n|\big\},              & \text{ pro }p= \infty\,.
    \end{cases} \]

\subsection*{Ortogonalita (kolmost)}

Nechť $\mathcal{H}$ je prostor se skalárním součinem a $\vx,\vy$ jsou vektory z
$\mathcal{H}$. Vektor $\vx$ nazýváme \textbf{ortogonální na} (nebo také
\textbf{kolmý} na) $\vy$, právě když

\[ \langle \vx\mid \vy\rangle =0\,. \]

\noindent Soubor vektorů $(\vx_1,\dots,\vx_n)$ z $\mathcal{H}$ nazveme
\textbf{ortogonální (OG)}, právě když každý vektor ze souboru je ortogonální na
ostatní vektory ze souboru, tj. pro každé $i,j\in\hat{n}$, $i\neq j$ je

\[ \langle \vx_{i}\mid \vx_{j}\rangle=0\,. \]

\noindent Soubor vektorů $(\vx_1,\dots,\vx_n)$ z $\mathcal{H}$ nazveme
\textbf{ortonormální (ON)}, právě když je ortogonální a každý vektor má
velikost $1$, tzn. pro každé $i,j\in\hat{n}$ Je

\[ \langle \vx_{i}\mid \vx_{j}\rangle=
    \begin{cases}
        0 & \text{ pro }i\neq j\,, \\
        1 & \text{ pro }i=j\,.
    \end{cases} \]

\subsection*{Pythagorova věta}

Nechť $\vx$ a $\vy$ jsou vektory z $\mathcal H$ a $\vx$ je kolmý na $\vy$.
Potom pro normu indukovanou skalárním součinem platí

\[ \|\vx+\vy\|^{2}=\|\vx\|^{2}+\|\vy\|^{2}\,. \]

\noindent Obecněji: Je-li $(\vx_1,\dots,\vx_n)$ ortogonální soubor z $\mathcal H$, potom

\[ \big\|\vx_1+\dots+\vx_n\big\|^{2}=\|\vx_1\|^{2}+\dots+\|\vx_n\|^{2}\,. \]

\subsection*{Věta o lineární nezávislosti OG souboru}

Ortogonální soubor \textbf{nenulových} vektorů je LN. Speciálně, každý
ortonormální soubor vektorů je LN.

\subsection*{Fourierovy koeficienty vůči ON bázi}

Nechť $\mathcal{X}=(\vx_1,\dots, \vx_n)$ je ON báze prehilbertova prostoru
$\mathcal{H}$, potom pro každé $\vz\in\mathcal{H}$ platí

\[ \vz=\sum_{i=1}^n \langle \vx_i\mid \vz\rangle \vx_i. \]

\noindent Neboli

\[ (\vz)_{\mathcal X}=\big(\langle \vx_1\mid \vz\rangle, \langle \vx_2\mid \vz\rangle , \dots ,\langle \vx_n\mid \vz\rangle\big)\,. \]

\subsection*{Ortogonální projekce na přímku}

Je-li $\vv\neq \theta$, zobrazení $\proj_\vv$ definované

\[ \proj_\vv(\vz):=\frac{\langle \vv\mid \vz\rangle}{\langle \vv\mid \vv\rangle} \vv \quad \text{ pro }\vz\in \mathcal{H} \]

\noindent se nazývá \textbf{ortogonální projekce $\vz$ na přímku} $\langle \vv\rangle$.

\subsection*{Gramova--Schmidtova ortogonalizace}

Nechť $\mathcal{X}=(\vx_1,\dots,\vx_n)\subseteq \mathcal{H}$ je LN soubor
vektorů. Potom existuje ON soubor $\mathcal{Y}=(\vy_1,\dots,\vy_n)$ vektorů z
$\mathcal{H}$ takový, že pro každé $k\in \hat{n}$ je

\[ \langle\vx_1,\dots,\vx_k\rangle    =\langle\vy_1,\dots,\vy_k\rangle\,. \]

\subsection*{Výpočet GSO}

\begin{align*}
    \vz_1 & =\vx_1\,,                                                                           \\
    \vz_2 & =\vx_2-
    \proj_{\vz_1}(\vx_2)\,,                                                                     \\
    \vz_3 & =\vx_3-\proj_{\vz_1}(\vx_3)-\proj_{\vz_2}(\vx_3)\,,                                 \\
          & \vdots                                                                              \\
    \vz_n & =\vx_n-\proj_{\vz_1}(\vx_n)-\proj_{\vz_2}(\vx_n)-\dots -\proj_{\vz_{n-1}}(\vx_n)\,.
\end{align*}

\pagebreak
