\section{Obecný vektorový prostor}

\subsection*{Grupa}

Nechť $M$ je neprázdná množina a $\circ: M \times M \to M$ binární operace.
Platí-li

\begin{enumerate}
      \item \textbf{asociativní zákon}: $(\forall a,b,c \in M) \big(a \circ (b \circ c) = (a \circ b) \circ c \big)$,
      \item existence \textbf{neutrálního prvku}: existuje $e \in M$ tak, že $(\forall a
                  \in M) \big( a \circ e = e \circ a = a \big)$,
      \item existence \textbf{inverzních prvků}: $(\forall a \in M)(\exists a^{-1}\in M)
                  \big(a \circ a^{-1} = a^{-1} \circ a = e\big)$,
\end{enumerate}

říkáme, že uspořádaná dvojice $G = (M, \circ)$ je \textbf{grupa}.

Platí-li navíc pro $\circ$

\begin{itemize}
      \item \textbf{komutativní zákon}: $(\forall a,b \in M) \big(a \circ b = b \circ a\big)$,
\end{itemize}

mluvíme o \textbf{abelovské grupě}.

\subsection*{Těleso}

Nechť $M$ je neprázdná množina a $+: M \times M \to M$, $\cdot : M \times M \to
      M$ dvě binární operace. Platí-li, že

\begin{enumerate}
      \item $(M, +)$ je \textbf{abelovská grupa} (neutrální prvek značíme $0$ a nazýváme nulovým prvkem),
      \item $(M\setminus \{0\}, \cdot)$ je grupa (neutrální prvek značíme $1$ a nazýváme jednotkový prvek),
      \item platí levý a pravý \textbf{distributivní zákon}, tj.

            \[ (\forall a,b,c \in M) \Big( a(b + c) = ab + ac \wedge (b+c)a = ba + ca\Big)\,, \]
\end{enumerate}

nazýváme uspořádanou trojici $T = (M, +, \cdot)$ \textbf{tělesem}.

Je-li navíc $(M\setminus \{0\}, \cdot)$ abelovská grupa, je $T$
\textbf{komutativní těleso}.

\subsection*{Vektorový prostor}

Nechť $T$ je libovolné komutativní těleso, jeho neutrální prvky vůči operacím
sčítání, resp. násobení označme $0$, resp. $1$. Mějme dánu neprázdnou množinu
$V$ a dvě zobrazení

\[ \oplus: V\times V\to V,\qquad \odot:T\times V\to V\,. \]

Řekneme, že $(V,T,\oplus,\odot)$ je \textbf{vektorový prostor nad tělesem} $T$ s vektorovými operacemi $\oplus$ a $\odot$, právě když platí následující \textbf{axiomy vektorového prostoru}:

\begin{enumerate}
      \item Sčítání vektorů je komutativní:

            \[ (\forall \vx,\vy\in V)(\vx\oplus \vy=\vy\oplus\vx)\,. \]

      \item Sčítání vektorů je asociativní:

            \[ (\forall \vx,\vy,\vz\in V)\big((\vx\oplus \vy)\oplus \vz=\vx\oplus(\vy\oplus \vz)\big)\,. \]

      \item Násobení skalárem je asociativní:

            \[ (\forall\alpha,\beta\in T)(\forall \vx\in V) \big(\alpha\odot(\beta\odot \vx)=(\alpha\cdot\beta)\odot \vx \big)\,. \]

      \item Násobení skalárem je distributivní zleva:

            \[ (\forall\alpha\in T) (\forall \vx,\vy\in V) \big(\alpha\odot(\vx\oplus \vy)=(\alpha\odot \vx)\oplus(\alpha\odot \vy)\big)\,. \]

      \item Násobení skalárem je distributivní zprava:

            \[ (\forall\alpha,\beta\in T)(\forall \vx\in V) \big((\alpha+\beta)\odot \vx=(\alpha\odot \vx)\oplus(\beta\odot \vx)\big)\,. \]

      \item Neutrální prvek $1\in T$ je neutrální i vůči násobení vektoru skalárem:

            \[ (\forall \vx\in V) (1\odot \vx=\vx)\,. \]

      \item Existuje \textbf{nulový vektor} ve $V$ a nulový násobek libovolného vektoru je
            nulový vektor

            \[ (\exists\theta\in V) (\forall \vx\in V) (0\odot \vx=\theta)\,.\]

\end{enumerate}

\subsection*{Základní vlastnosti VP}

Buď $V$ vektorový prostor nad tělesem $T$. Potom platí:

\begin{enumerate}
      \item Ve $V$ existuje právě jeden nulový vektor.
      \item Libovolný násobek nulového vektoru je opět nulový vektor. Tj.
            \[ (\forall \alpha\in T) (\alpha\odot\theta=\theta)\,. \]
      \item Přičtení nulového vektoru k libovolnému vektoru jej nezmění. Tj.
            \[ (\forall \vx\in V) (\vx \oplus \theta=\vx)\,. \]

      \item Ke každému vektoru z $V$ existuje právě jeden \textbf{vektor opačný}. Tzn.,

            \[ (\forall \vx\in V)(\exists_{1} \vy\in V)( \vx\oplus\vy=\theta)\,. \]
            Tento vektor splňuje $\vy= (-1)\odot \vx$, kde $-1$ je opačný prvek k $1$ vůči
            operaci $+$ v $T$.
      \item Je-li součin skaláru a vektoru roven nulovému vektoru, potom je skalár roven
            $0$ nebo vektor roven $\theta$.

            \[ (\forall \alpha \in T) (\forall \vx \in V)  \Big(\alpha\odot  \vx=\theta \Rightarrow (\alpha=0 \vee \vx=\theta)\Big)\,. \]
\end{enumerate}

\subsection*{Podprostor}

Nechť $P$ je podmnožina vektorového prostoru $\color{blue}{V}$ nad $T$.
Řekneme, že $P$ je \textbf{podprostor} vektorového prostoru $\color{blue}{V}$,
právě když platí:

\begin{enumerate}
      \item množina $P$ je neprázdná, tzn. $P\neq \emptyset.$
      \item množina $P$ je \emph{uzavřená} vůči sčítání vektorů v ní, tzn.

            \[ (\forall \vx,\vy\in P)(\vx+\vy\in P)\,, \]

      \item Množina $P$ je uzavřená vůči násobení vektorů v ní libovolným skalárem, tzn.

            \[ (\forall\alpha\in T)(\forall \vx\in P) (\alpha \vx\in P)\,. \]

\end{enumerate}

Vztah být podprostorem pak značíme

\[ P\subset \subset \color{blue}{V}. \]

\subsection*{(Triviální) lineární kombinace}

Mějme vektorový prostor $\color{blue}{V}$ nad $T$. Nechť $\vx\in
      \color{blue}{V}$ a $(\vx_{1},\dots,\vx_{m})$ je soubor vektorů z
$\color{blue}{V}$. Říkáme, že vektor $\vx$ je \textbf{lineární kombinací}
souboru $(\vx_{1},\dots,\vx_{m})$, právě když existují čísla
$\alpha_{1},\dots,\alpha_{m}\in T$ taková, že

\[ \vx=\sum_{i=1}^{m}\alpha_{i}\vx_{i}\,. \]

Čísla $\alpha_{i}$, $i\in\hat{m}$, nazýváme \textbf{koeficienty lineární kombinace}.
Jestliže $(\forall i\in\hat{m})(\alpha_{i}=0)$, nazýváme takovou lineární kombinaci \textbf{triviální}.
V opačném případě jde o \textbf{lineární kombinaci netriviální}.

\subsection*{Lineárně (ne)závislý soubor}

\begin{itemize}
      \item $(\vx_{1},\dots,\vx_{m})$ je LN $\Leftrightarrow$
            \[ (\forall \alpha_{1},\dots,\alpha_{m}\in T)\left(\,\sum_{i=1}^{m}\alpha_{i}\vx_{i}=\theta \Rightarrow \big((\forall i\in\hat{m})(\alpha_{i}=0)\big)\,\right)\,. \]
      \item $(\vx_{1},\dots,\vx_{m})$ je LZ $\Leftrightarrow$
            \[ (\exists \alpha_{1},\dots,\alpha_{m}\in T) (\exists k\in\hat{m})
                  \left(\,\alpha_{k}\neq0 \wedge \,\sum_{i=1}^{m}\alpha_{i}\vx_{i}=\theta\,\right)\,.
            \]
\end{itemize}

\subsection*{Lineární obal souboru}

Buď $(\vx_{1},\dots,\vx_{m})$ soubor vektorů z VP $V$ nad tělesem $T$. Množinu
všech lineárních kombinací tohoto souboru nazveme \textbf{lineárním obalem
      souboru} $(\vx_{1},\dots,\vx_{m})$ a značíme ji

\[ \langle  \vx_{1},\dots,\vx_{m}  \rangle . \]

\noindent Neboli

\[ \langle  \vx_{1},\dots,\vx_{m}  \rangle = \left\{\sum_{i=1}^m \alpha_i \vx_i \ \Big|\ \alpha_i\in T \text{ pro každé }i \in \hat m \right\}\,. \]

\subsection*{Vlastnosti lineární obalu souboru}

Nechť $(\vx_1,\ldots,\vx_m)$ je soubor vektorů z vektorového prostoru $V$ nad
$T$. Pak platí:

\begin{enumerate}
      \item lineární obal obsahuje nulový vektor:
            \[ \theta\in \langle \vx_1,\ldots,\vx_m \rangle , \]
      \item vektory leží ve svém lineárním obalu, přesněji:
            \[ \vx_1,\ldots,\vx_m  \in  \langle \vx_1,\ldots,\vx_m \rangle\,, \]
      \item je-li vektor již obsažen v lineárním obalu, tak jeho přidáním do souboru se
            lineární obal nezmění:

            \[ (\forall \vz\in V)\Big(\vz\in\langle \vx_1,\ldots,\vx_m \rangle   \quad \Leftrightarrow \quad \langle  \vx_1,\ldots,\vx_m  \rangle =\langle  \vx_1,\ldots,\vx_m,\vz  \rangle\Big)\,, \]

      \item lineární obal je uzavřený na sčítání vektorů i na násobení vektorů skalárem:
            \begin{gather*}
                  (\forall\vx,\vy\in\langle  \vx_1,\ldots,\vx_m \rangle)\big(\vx+\vy\in\langle \vx_1,\ldots,\vx_m \rangle\big)\\
                  \quad\text{a}\quad\\
                  (\forall\alpha\in T)(\forall\vx\in\langle \vx_1,\ldots,\vx_m \rangle)\big(\alpha \vx\in\langle \vx_1,\ldots,\vx_m \rangle\big) .
            \end{gather*}
      \item lineární obal z lineárního obalu neobsahuje nic navíc:
            \begin{gather*}
                  \text{Je-li } k\in \N \text{ a } \vz_1,\dots, \vz_k\in\langle  \vx_1,\ldots,\vx_m \rangle\,,\text{ potom }\\
                  \langle  \vz_1,\ldots,\vz_k  \rangle \text{ je podmnožinou } \langle  \vx_1,\ldots,\vx_m \rangle\,.
            \end{gather*}
\end{enumerate}

\subsection*{Lineární obal množiny}

Buď $M$ neprázdná podmnožina VP $V$ nad tělesem $T$. Množinu všech lineárních
kombinací všech souborů vektorů z množiny $M$ nazveme \textbf{lineárním obalem
      množiny} $M$ a značíme ji $\langle M \rangle $.

Tedy

\[ \langle  M \rangle
      =\left\{\sum_{i=1}^{m}\alpha_{i}\vx_{i}\ \Big|\ m\in \N,\ \vx_1,\dots,\vx_m\in M,\ \alpha_1,\dots,\alpha_m\in T\right\}\,. \]

\subsection*{Vlastnosti lineárního obalu množiny}

Nechť $M$ je neprázdná množina vektorů z vektorového prostoru $V$ nad $T$. Pak
platí:

\begin{enumerate}
      \item lineární obal obsahuje nulový vektor:
            \[ \theta\in \langle M \rangle , \]
      \item vektory z $M$ leží v jeho lineárním obalu, přesněji:
            \[ M \subseteq \langle M \rangle\,, \]
      \item je-li vektor již obsažen v lineárním obalu, tak jeho přidáním do množiny se
            lineární obal nezmění:
            \[ (\forall \vz\in V)\Big(\vz\in\langle M \rangle    \Leftrightarrow \langle  M  \rangle =\big\langle  M\cup\{\vz\}  \big\rangle\Big)\,, \]
      \item lineární obal je uzavřený na sčítání vektorů i na násobení vektorů skalárem:
            \begin{gather*}
                  \big(\forall\vx,\vy\in\langle  M \rangle\big)\big(\vx+\vy\in\langle M \rangle\big)\\
                  \quad\text{a}\quad\\
                  (\forall\alpha\in T)\big(\forall\vx\in\langle M \rangle\big)\big(\alpha \vx\in\langle M \rangle\big) .
            \end{gather*}
      \item lineární obal souboru vektorů z lineárního obalu souboru vektorů neobsahuje nic
            navíc:
            \begin{gather*}
                  \text{Je-li } \emptyset \neq N\subseteq \langle  M \rangle\,,\text{ potom }\\
                  \langle  N  \rangle \subseteq \langle  M \rangle\,.
            \end{gather*}
            Speciálně:
            \[ \text{Je-li } \emptyset \neq N\subseteq M,\text{ potom }\\
                  \langle  N  \rangle \subseteq \langle  M \rangle\,. \]
\end{enumerate}

\subsection*{Věta o vztahu LZ souboru a lineárního obalu}

Buď $(\vx_{1},\dots,\vx_{m})$ soubor vektorů z VP $V$ a $m\geq 2$. Potom
$(\vx_{1},\dots,\vx_{m})$ je lineárně závislý právě tehdy, když

\[ (\exists k\in\hat{m})\big(\vx_{k}\in\langle \vx_{1},\dots,\vx_{k-1},\vx_{k+1},\dots,\vx_{m} \rangle\big) \,. \]

\subsection*{Přidání vektoru do LN souboru}

Buď $(\vx_{1},\dots,\vx_{m})$ LN soubor vektorů z VP $V$ a $\vy\notin\langle
      \vx_{1},\dots,\vx_{m} \rangle $. Potom soubor $(\vx_{1},\dots,\vx_{m},\vy)$ je
také LN.

\subsection*{Soubor (množina) generuje podprostor}

O souboru vektorů $(\vx_1,\dots,\vx_m)$ z vektorového prostoru $V$ řekneme, že
\textbf{generuje} podprostor $P\ssubset V$, právě když platí:

\[ \langle \vx_1,\dots,\vx_m \rangle =P. \]

V případě, že $P=V$ můžeme zjednodušeně říkat $(\vx_1,\dots,\vx_m)$ generuje
(vektorový) prostor $V$.

