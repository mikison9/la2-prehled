\section{Lineární zobrazení}

\subsection*{Lineární zobrazení}

Buďte $P$ a $Q$ dva vektorové prostory nad stejným tělesem $T$. Zobrazení
$A:P\rightarrow Q$ nazveme \textbf{lineární}, právě když současně platí:

\begin{enumerate}
      \item (aditivita):

            \begin{equation*}
                  (\forall \vx,\vy\in P)( A(\vx+\vy)=A\vx+A\vy)\,,
            \end{equation*}
      \item (homogenita):

            \begin{equation*}
                  (\forall \alpha\in T)( \forall \vx \in P) \big(A(\alpha \vx)=\alpha A\vx\big)\,.
            \end{equation*}
\end{enumerate}

\subsection*{Lineární operátor (transformace), funkcionál a izomorfismus}

Lineární zobrazení z VP $P$ do stejného VP $P$ nazýváme \textbf{lineární
      operátor} (nebo také \textbf{transformace}) na $P$.

\vspace{0.5em}

\noindent Množinu všech lineárních operátorů na $P$ značíme krátce $\mathcal L(P)$.

\vspace{0.5em}

\noindent Lineární zobrazení z VP $P$ do tělesa $T$ nazýváme \textbf{lineární funkcionál} na $P$.

\vspace{0.5em}

\noindent \textbf{Izomorfismem} nazveme lineární zobrazení, které je zároveň i bijekce.

\subsection*{Souřadnicový izomorfismus}

Nechť soubor $\mathcal B=(\vx_1,\dots,\vx_n)$ je báze prostoru $V$ nad $T$.
Přiřazení $(\cdot)_{\mathcal B}:V\to T^n$ definované předpisem

\[ \vx\mapsto (\vx)_{\mathcal B} \quad \text{pro } \vx\in V, \]

\noindent kde $(\vx)_{\mathcal B}$ značí souřadnice vektoru $\vx$ vůči bázi $\mathcal B$
dle Definice .reference:dfn-souradnice, nazýváme \textbf{souřadnicový
      izomorfismus}.

\subsection*{Alternativní vyjádření linearity}

Buďte $P$ a $Q$ vektorové prostory nad $T$ a mějme zobrazení $A:P\rightarrow
      Q$. Následující tři tvrzení jsou ekvivalentní:

\begin{enumerate}
      \item $A\in\mathcal L(P,Q)$.
      \item $(\forall \alpha\in T)(\forall \vx,\vy\in P)$

            \[ \left(A(\alpha \vx+ \vy)=\alpha A\vx+ A\vy\right)\,. \]

      \item $(\forall n\in\mathbb{N})(\forall \alpha_{1},\dots,\alpha_{n}\in T)(\forall \vx_{1},\dots, \vx_{n}\in P)$

            \[ \bigg(A\left(\sum_{i=1}^{n}\alpha_{i}\vx_{i}\right)=\sum_{i=1}^{n}\alpha_{i}A\vx_{i}\bigg)\,. \]

\end{enumerate}

\subsection*{Věta o inverzi a skládání lineárních zobrazení}

Mějme $P,Q,R$ vektorové prostory nad stejným tělesem. Potom:

\begin{enumerate}
      \item Buďte $A\in\mathcal L(P,Q)$ a $B\in\mathcal L(Q,R)$. Potom složené zobrazení
            $BA$ je také lineární, tj.

            \[ BA\in\mathcal L(P,R)\,. \]

      \item Je-li $A\in\mathcal L(P,Q)$ izomorfismus"Neboli toto lineární zobrazení je
            navíc bijekce, tedy k němu existuje inverzní zobrazení.".footnote, potom
            inverzní zobrazení $A^{-1}$ je také lineární, tj.

            \[ A^{-1}\in\mathcal L(Q,P)\,. \]
\end{enumerate}

\subsection*{Věta o určení lineárního zobrazení pomocí obrazů báze}

Nechť $P$, $Q$ jsou vektorové prostory nad $T$. Nechť $(\vx_1,\dots,\vx_n)$ je
báze $P$ a nechť $(\vy_1,\dots,\vy_n)$ je libovolný soubor vektorů z $Q$. Potom
existuje právě jedno lineární zobrazení $A\in\mathcal L(P,Q)$ takové, že

\[ (\forall i\in\hat{n})(A\vx_{i}=\vy_{i})\,. \]

\subsection*{Základní vlastnosti lineárního zobrazení}

Nechť $A\in\mathcal L(P,Q)$, kde $P,Q$ jsou vektorové prostory nad $T$.

\begin{enumerate}
      \item \emph{Obraz nulového vektoru je nulový vektor:}  Označíme-li nulové vektory v $P$ a $Q$ popořadě $\theta_P$ a $\theta_Q$, platí

            \[ A\theta_P=\theta_Q\,. \]

      \item \emph{Obraz lineárního obalu je lineární obal obrazu:} Je-li $M\subseteq P$, potom

            \[ A\big(\langle M\rangle \big)=\big\langle A(M)\big\rangle \,. \]

            Je-li $(\vx_1,\dots,\vx_n)$ soubor vektorů z $P$ , potom

            \[ A\big(\langle \vx_1,\dots,\vx_n\rangle \big)=\langle A\vx_1,\dots,A\vx_n\rangle \,. \]

      \item \emph{Obraz podprostoru je podprostor.} Neboli

            \[ \big(\forall \tilde{P}\ssubset P\big) \big(A(\tilde{P})\ssubset Q\big)\,. \]

      \item \emph{Vzor podprostoru je podprostor.} Neboli

            \[ \big(\forall \tilde{Q}\ssubset Q\big) \big(A^{-1}(\tilde{Q})\ssubset P\big)\,. \]

      \item \emph{Obraz LZ souboru je opět LZ soubor.} Neboli pro libovolný soubor $(\vx_1,\dots,\vx_n)$ platí

            \[ (\vx_1,\dots,\vx_n)\text{ je } LZ \implies (A\vx_1,\dots,A\vx_n)\text{ je } LZ. \]

      \item \emph{``Předobraz'' LN souboru je opět LN soubor.} Přesněji pro libovolný soubor $(\vx_1,\dots,\vx_n)$ platí

            \[ (A\vx_1,\dots,A\vx_n)\text{ je } LN \implies (\vx_1,\dots,\vx_n)\text{ je } LN. \]

\end{enumerate}

\subsection*{Hodnost, jádro a defekt}

Nechť $A\in\mathcal L(P,Q)$. \textbf{Hodností zobrazení} $A$ rozumíme číslo

\[ h(A):=\dim A(P)\,. \]

\noindent \textbf{Jádro zobrazení} $A$ definujeme jako množinu

\[ \ker A:=\{\vx\in P\mid A\vx=\theta_Q\}\,, \]

\noindent a jeho dimenzi nazýváme \textbf{defektem zobrazení} $A$. Defekt značíme $d(A)$. Tedy

\[ d(A):=\dim \ker A\,. \]

\subsection*{2. věta o dimenzi}

Nechť $A\in\mathcal L(P,Q)$. Potom
\[ h(A)+d(A)=\dim P. \]

\subsection*{Věta o vztahu in/sur-jektivity a defektu/hodnosti}

Nechť $A\in\mathcal L(P,Q)$ a dimenze $\dim P$ a $\dim Q$ jsou konečné.

\begin{enumerate}
      \item \begin{gather*}
                  A \text{ je injektivní } \ \Leftrightarrow\ \ker A=\{\theta_P\}\\
                  \Leftrightarrow\  d(A)=0 \ \Leftrightarrow\  h(A)=\dim P,
            \end{gather*}
      \item \begin{gather*}
                  A \text{ je surjektivní }\ \Leftrightarrow\ A(P)=Q \\
                  \Leftrightarrow\ \dim A(P)=\dim Q \ \Leftrightarrow\ h(A)=\dim Q.
            \end{gather*}
\end{enumerate}

\subsection*{Věta o jádru prostého zobrazení}

Nechť $A\in\mathcal L(P,Q)$. Potom platí:

\[ A\text{ je prosté }\Leftrightarrow \ker A=\{\theta_P\}\,. \]

\subsection*{LN/LZ soubor a prosté zobrazení}

Nechť $A\in\mathcal L(P,Q)$ je \textbf{prosté}. Potom

\begin{enumerate}
      \item Obraz LN souboru vektorů je taky LN soubor. Tedy je-li $(\vx_1,\dots,\vx_n)$ LN
            soubor vektorů z $P$, je také $(A\vx_1,\dots,A\vx_n)$ LN.
      \item Vzor LZ souboru vektorů je opět LZ. Neboli: je-li $(\vx_1,\dots,\vx_n)$ soubor
            vektorů z $P$ takový že $(A\vx_1,\dots,A\vx_n)$ je LZ soubor, potom také soubor
            vzorů $(\vx_1,\dots,\vx_n)$ LZ.
\end{enumerate}

\subsection*{Lineární zobrazení a zachování dimenze}

Mějme $A\in \mathcal L(P,Q)$ a $M$ podprostor VP $P$. Potom

\begin{enumerate}
      \item $\dim A(M) \leq \dim M$.
      \item Je-li $A$ prosté zobrazení, potom $\dim A(M) = \dim M$.
      \item $h(A)\leq \min \{\dim P, \dim Q\}\,.$
\end{enumerate}

\pagebreak
