\section{Základy numerické matematiky}

\subsection*{Přidružená (indukovaná) maticová norma}

Mějme normu $\| \cdot \|_{(n)}$ na $\R^n$ a normu $\| \cdot \|_{(m)}$ na
$\R^m$, definujeme \textbf{přidruženou (indukovanou) maticovou normu} matice
$\mA \in \R^{m,n}$ následujícím způsobem

\begin{equation*}
	\| \mA\| := \sup_{\substack{\vx \in \R^{n} \\ \vx \neq \theta}} \frac{ \|\mA\vx\|_{(m)} }{\|\vx\|_{(n)}}\,.
\end{equation*}

\subsection*{O základních vlastnostech přidružené normy}

Zobrazení definované v předchozí definici je normou a platí pro ni

\begin{enumerate}
	\item Je-li $m=n$ a zvolené normy jsou stejné, potom $\|\mE\| = 1$ (zde $\mE$ je
	      jednotková matice),
	\item $\|\mA\vx\|_{(m)} \leq \|\mA\| \cdot \|\vx\|_{(n)}$ (konzistence normy),
	\item $\|\mA\mB\| \leq \|\mA\| \cdot \|\mB\|$.
\end{enumerate}

\subsection*{(Přidružená) maticová $p$-norma}

Pokud v definici maticové normy uvažujeme na $\R^n$ i $\R^m$ odpovídající
$p$-normy, nazýváme tuto normu přidruženou (indukovanou) \textbf{maticovou
	$p$-normou} a značíme ji $\|\mA\|_p$. Tedy
\[\| \mA\|_p = \sup_{\substack{\vx \in \R^{n} \\ \vx \neq \theta}}  \frac{\left\|\mA \vx\right\|_{p} }{\|\vx\|_{p}}
	= \sup_{\substack{\vz \in \R^{n} \\ \|\vz\|_{p} = 1}} \|\mA\vz\|_{p}.\]

\subsection*{Věta o speciálních případech přidružené maticové normy}

Mějme matici $\mA$ z $\R^{m,n}$, která má složky $a_{ij}$, potom

\begin{enumerate}
	\item Norma $\| \mA \|_{1}$ je rovna maximu součtu absolutních hodnot ve sloupci, tj.
	      \[ \| \mA \|_{1} = \max_{1 \leq j \leq n} \sum_{i = 1}^m |a_{i,j}|\,. \]
	\item Norma $\| \mA \|_{\infty}$ je rovna maximu součtu absolutních hodnot v řádku,
	      tj.
	      \[ \| \mA \|_{\infty} = \max_{1 \leq i \leq m} \sum_{j = 1}^n |a_{i,j}|\,. \]
	\item Norma $\| \mA \|_{2}$ je rovna odmocnině z největšího vlastního čísla matice
	      $\mA^T\mA$ i matice $\mA\mA^T$.
\end{enumerate}

\subsection*{Frobeniova norma}

\textbf{Frobeniovou normou} nazýváme normu na $\R^{m,n}$ definovanou předpisem
\[ \| \mA \|_{F} = \left( \sum_{i=1}^{m}  \sum_{j=1}^{n} a_{ij}^2 \right)^\frac{1}{2}. \]

\subsection*{Absolutní číslo podmíněnosti}

\textbf{Absolutní číslo podmíněnosti} úlohy je definováno

\[ \hat\kappa = \lim_{r \to 0+} \sup_{ \| \Delta x \| \leq r} \dfrac{ \| \Delta y \| }{ \| \Delta x \| }. \]

\subsection*{Relativní číslo podmíněnosti}

\textbf{Relativní číslo podmíněnosti} úlohy je
\[ \kappa = \lim_{r \to 0+} \sup_{ \| \Delta x \| \leq r}
	\left({ \dfrac {\| \Delta y \| }{\|y\|}} \Big/ { \dfrac{ \|\Delta x\| }{\|x\|}}\right). \]

\subsection*{Číslo podmíněnosti matice}

Mějme regulární matici $\mA \in \R^{n,n}$. Číslo
\[ \kappa(\mA) = \|\mA\| \cdot \| \mA^{-1} \| \]
se nazývá \textbf{číslo podmíněnosti matice} $\mA$ vhledem k normě $\|\cdot\|$.

\subsection*{Věta o podmíněnosti řešení soustavy lineárních rovnic}

Mějme matici $\mA \in \R^{n,n}$. Změníme-li vektor pravé strany soustavy
lineárních rovnic $\vb$ o $\Delta \vb$, pro změnu řešení $\Delta \vx$ platí
\[ \frac{\| \Delta \vx \|}{\| \vx \|} \leq \kappa(\mA) \frac{\| \Delta \vb \|}{\| \vb \|}. \]

\subsection*{Stabilita algoritmů}

Nechť $V$ je nějaký numerický algoritmus, jehož teoretický (přesný) výstup
označíme $V^*(x)$, kde $x$ jsou vstupní data. Výsledek výpočtu v konečné
(strojové) aritmetice označíme $V(x)$. Označme $\Delta y = V^*(x) - V(x)$.

Tato hodnota je tzv. \emph{dopředná/přímá chyba} (anglicky \emph{forward
	error}). Je to odchylka spočítaného řešení od přesného řešení.

Nejmenší (v normě) číslo $\Delta x$ takové, že $V^*(x + \Delta x) = V(x)$ se
nazývá \emph{zpětná chyba} (anglicky \emph{backward error}). Jedná se o
promítnutí chyby algoritmu $V$ do jeho vstupu.

\pagebreak